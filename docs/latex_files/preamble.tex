% Hinweis: Optionen der Dokumentenklasse werden an alle folgenden \usepackage{package} Befehle weitergegeben
\documentclass[fontsize=12pt,paper=a4,parskip=half,twoside=false,numbers=noenddot,bibliography=totoc,listof=totoc,titlepage=true,headsepline=true,footsepline=true]{scrreprt}


% Zeichenkodierung Input ist UTF-8: Umlaute können direkt eingegeben werden
\usepackage[utf8]{inputenc}

% Zeichenkodierung Ausgabe ist T1-Kodierung: Wichtig für die Ausgabe von Umlauten
\usepackage[T1]{fontenc}

% Schrift festlegen
\usepackage{lmodern}

% Sprachauswahl für Lokalisierungen und Silbentrennung
\usepackage[english]{babel}

% Zitate: Anführungszeichen automatisch anhand der Sprache wählen
\usepackage[babel=true]{csquotes}

% Seitenränder setzen
\usepackage[left=2cm, right=3.5cm, top=2.5cm, bottom=2cm, headheight=1.25cm, footskip=1.25cm]{geometry}

%mehrere Spalten
\usepackage{multicol}

% Zeilenabstand auf 1.5 setzen
\usepackage{setspace}
\onehalfspacing

% Paket zum Anpassen von Kopf- und Fuss?zeilen
\usepackage[plainfootsepline, plainheadsepline, headsepline, footsepline, automark]{scrpage2}

% Liniendicke
\setheadsepline{0.1pt}
\setfootsepline{0.1pt}

% Kopf- und Fusszeile loeschen
\clearscrheadfoot
% Kopf- und Fusszeile aktivieren
\pagestyle{scrheadings}

% Kopf links
\ihead[\headmark]{\headmark}

% Fuss rechts
\ofoot[\pagemark]{\pagemark}

% Seitenzahlen fuer grosse roemische Zahlen
\newcounter{RomanPagenumber}

%symbols for bibliography
\usepackage{url}
\usepackage[backend=bibtex, style=alphabetic, url=true]{biblatex}
\DeclareNameAlias{author}{last-first}
\bibliography{bib/literature}
\usepackage{booktabs}
\renewcommand{\arraystretch}{1.2}

% Grafiken einbinden
\usepackage{graphicx}
\graphicspath{{./fig/}}

%Durchgehende Nummerierung
\usepackage{chngcntr}
\counterwithout{figure}{chapter}
\counterwithout{table}{chapter}

%tabellen
\usepackage{multirow}
\usepackage{graphicx}
\usepackage{tabularx}
\usepackage{adjustbox}

%Algorithms
\usepackage{algorithm}
\usepackage{algpseudocode}

%Definitionen
\usepackage{amsthm}
\usepackage{mathrsfs}
\usepackage{amsmath,amssymb,amstext}
\newtheorem{definition}{Definition}%[section]
\newtheorem{theorem}{Theorem}%[section]
\DeclareMathOperator*{\argmax}{arg\,max}
\DeclareMathOperator*{\argmin}{arg\,min}

%custom commands
\newcommand{\titel}{The Effect of Training Data Correlation on Differentially Private Neural Networks}

%mehere bilder nebeneinander
\usepackage{subcaption}

% Code listings
\usepackage{listings}
\lstset{basicstyle=\ttfamily}

% Links- und PDF-Einstellungen
\usepackage{hyperref}
\hypersetup{pdftitle = {\titel},pdfkeywords = {},pdfstartview = {Fit},	colorlinks = {false},breaklinks = {true},bookmarksopen = {true}}

% Verhinderung das Absaetze mit einzeilern beginnen & enden 
\clubpenalty = 10000
\widowpenalty = 10000
\displaywidowpenalty = 10000

%title formatierung
\usepackage{longtable}

%abkuerzungen
\usepackage{acronym}

%Title
\renewcommand*{\chapterheadstartvskip}{\vspace*{0.7cm}}

%Umbrueche erlauben
\allowdisplaybreaks

%eigenes TODO und Comment command
\usepackage{todonotes}

%PDF import
\usepackage[final]{pdfpages}

%comment out big tex
\usepackage{blindtext, xcolor}
\usepackage{comment}

%picture location settings (avoid floating)
\usepackage{float}

%appendix
\usepackage{appendix}


